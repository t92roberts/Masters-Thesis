% Chapter Template

\chapter{Background}\label{ChapterBackground}

This chapter gives an overview of the two important themes of this thesis: agile software development and optimisation problems. It will establish the vocabulary used throughout the rest of the paper and outline the concepts about these two topics that will give some context to the implementation and results of the work.

\section{Agile Software Development}

When writing software became less of an academic exercise and more of a commercial one, there were few frameworks available for how to manage this process. \citet{benington1983production} is cited as one of the first to suggest an approach to developing large programs inspired by the highly-structured processes used in manufacturing. It consisted of a nine stage plan that maps out the phases of a project, starting with creating a list of requirements for what the system needs to do, implementation, testing and end-user evaluation. \citet{royce1987managing} was one of the first to formalise this process and it would later become known as the Waterfall Model (notice that Royce used this model as an example of a bad way to develop software that should be avoided). Since then, many frameworks have been proposed as an alternative to waterfall that try to deliver software products on time, within budget, but most importantly, that meets the customer's expectations.

Agile software development is one such approach that has been widely adopted by development teams. Agile is the term for a collection of frameworks based on the values described in the Manifesto for Agile Software Development (\citet{beck2001manifesto}). Agile methodologies focus not only on how to organise and manage the technical work required to deliver a product, but also how to foster cohesion and collaboration both within the development team and in the organisation. Broadly, Agile methodologies aim to increase the quality of software deliveries by putting great emphasis on building and maintaining a well-functioning team that is able to deliver the functionality that the customer values the most. Empirical studies such as \citet{dybaa2008empirical} have shown that depending on the size and maturity of the development team (and the company), it can sometimes be difficult to adopt the process. However once it is adopted, it is generally seen as a positive step in managing the software development process. In particular, the human and social factors that the principles foster can create a team that is more relaxed and collaborative.

%give a summary of Scrum and the development cycle (use diagrams)

A central part of Agile is the idea of working in \emph{sprints} -- short periods of time where the development team works on tasks (known as \emph{user stories}) that deliver the most value to the customer at that moment in time. A sprint typically lasts 1 or 2 weeks but can be much shorter. Working to these very short time horizons instead of long project cycles means that if the customer's priorities change during a project, unforeseen problems occur, or entirely new requirements are added, the agile development team is able to shift focus after every sprint to accommodate this. The main roles in an agile team are the \emph{Product Owner}, the \emph{Development Team}, and the \emph{Scrum Master}.

The Product Owner may either be the customer themselves or a representative for a group of customers. They are responsible for digesting the sometimes conflicting needs of the customers and maintaining a \emph{Product Backlog} that acts as a wish list of functionality that the customers want in the product. Any item in the backlog has a chance (but not a certainty) of being implemented, and anything not in the backlog will not be implemented. Every item is given a \emph{business value} which is a relative value of importance, and the Product Owner is responsible for ensuring that the most important tasks are given top priority. They are also responsible for writing user stories that the Development Team can understand so that it is easy for them to implement and test. A clear 'Definition of Done' is essential so that the Product Owner and Development Team can agree when a user story has been delivered successfully.

The Development Team is responsible for carrying out the technical work to deliver working software. They are empowered to estimate how much effort it will take to implement each task in the Product Backlog. There has been much debate and advice about how to effectively estimate the effort of user stories \citep{cohn2004user} and the measure used in this thesis is \emph{story points}. Every agile team is free to define their own scale of story points but typically they are not equivalent to elapsed time. Instead, they indicate a combination of how difficult, uncertain, and large a task is. The team can use techniques such as Planning Poker \citep{cohn_planning_poker} to make realistic estimates that everyone agrees on. Together with the Product Owner, the Development Team may also add more user stories to the backlog, for example, if a user story is dependent on some other work to be finished before it can be started on. If a user story is judged to be too large to fit into one sprint, they can work with the Product Owner to redefine the scope of the Definition of Done and break it down into multiple smaller stories that can each be delivered within a sprint. The Development Team is also responsible for communicating their planned vacation and allocation during a sprint so that the team can plan its work.

The Scrum Master is responsible for encouraging all members of the team to work in line with the principles of the Agile Manifesto and generally try to ensure that the Development Team runs efficiently and is able to deliver what the Product Owner is asking for. They work to remove any obstacles or problems blocking the Development Team from working on the tasks they have committed to. Before a sprint starts, the team holds a sprint planning meeting where they decide on which user stories will be taken in to the coming sprint. When planning the upcoming sprint, the plan should strike a good balance between delivering high-value user stories and not overloading the Development Team. The Product Owner refines the business value estimations, the Development Team refines the story point estimations, and the Scrum Master facilitates both parties to agree on a realistic and valuable sprint plan.

\section{Combinatorial Optimisation Problems}

Combinatorial Optimisation (CO) is an area of computer science where a problem is modelled in a way that a solution can be expressed as a set of assignments of values to variables. An objective function can then take a solution as an input and return a value that indicates how good the solution is. The goal is to search the set of all possible solutions to find the optimal set of assignments that maximises or minimises the objective function (depending on what is desirable for a particular problem). Notable examples of CO problems are the Travelling Salesman Problem (TSP), the Quadratic Assignment Problem (QAP), and the Resource-Constrained Project Scheduling Problem (RCPSP). Because many real-world problems can be modelled as a CO problem, there has been much research into how to solve them. Two broad classes of approaches have emerged: complete algorithms and approximation algorithms.

\subsection{Complete Algorithms}

The set of all possible solutions to a CO problem is typically finite, so a complete algorithm that can exhaustively search every possible solution is guaranteed to find the optimal one. However, CO problems that are NP-hard \citep{garey1979computers} may take an exponentially-increasing time to solve which is impractical in most cases. The Branch and Bound algorithm \citep{land1960automatic} is the most widely-used tool for solving large-scale NP-hard combinatorial optimisation problems \citep{clausen1999branch}. It represents the solution space as a rooted tree where the root node represents a solution where none of the variables have been assigned, a node in the tree represents a partial solution, and a leaf represents a complete solution. As the algorithm traverses the tree, it maintains an upper bound on the cost function over all of the solutions it has explored so far. If a new solution has a cost above the upper bound, the branch is 'cut' because it is impossible to find a lower-cost solution in the rest of the branch. This can dramatically reduce the number of solutions that are explored while still guaranteeing to find the optimal solution. The basic algorithm starts with a pessimistic upper bound of $\infty$ but this bound is tightened as the search progresses. However in the worst case, the bound does not get close to the optimal cost so few cuts are made and the search degenerates to an exhaustive search.

\section{Approximation Algorithms}
Many real applications do not necessarily require the optimal solution to a problem and a solution that is close to optimal is sufficient. By relaxing the requirement of finding the optimal solution, approximation algorithms that converge towards the optimal but may not reach it can find a very good solution in much less time than a complete algorithm.

Constructive algorithms start with an empty or partial solution and try to make assignments until a good solution is found. These can run very fast but often give poor results compared to local search algorithms \citep{blum2003metaheuristics}. Local search algorithms start with a complete solution and try to modify some of its assignments to improve the value of the solution. Making small changes to a solution to create new but similar solutions is known as generating its \emph{neighbourhood}. A local search algorithm explores a solution's neighbourhood by generating its neighbours, comparing their values, and selecting the best neighbour to accept as the new 'current' solution. Ideally, this process can be repeated and the search follows a path of increasingly-valuable solutions until it reaches the global optimum. Local search has the additional benefit that it does not need to maintain a search tree and only one solution needs to be stored in memory, making it very memory-efficient.

As described in \citet{ahuja2002survey}, a neighbourhood search algorithm can be defined as: (1) a neighbourhood graph where a node represents a feasible solution and a directed arc $(S,T)$ indicates that $T$ is in the neighbourhood of $S$, (2) a method for searching the neighbourhood graph at each iteration, (3) a method for deciding which is the next node that the search will choose. In general, the larger the neighbourhood, the better. The local search is able to 'look' further away from its current solution and reduce its chance of becoming stuck in a small area of the solution space. However, some neighbourhoods may be very large and generating, evaluating and comparing lots of neighbours may itself take a long time and increase the overall time of the local search.

\section{Metaheuristics}

Metaheuristics are high-level strategies that are not related to a specific problem and can try to guide a local search through the solution space. Well-known metaheuristics include Hill Climbing, Random Restarts, Simulated Annealing, Tabu Search, Genetic Algorithms, Evolutionary Algorithms, and Constraint-based Local Search. Any of these techniques can be used alone or in combination with other metaheuristics to help avoid the common pitfalls of local search.

\subsection{Hill Climbing}

Steepest-descent hill-climbing search is perhaps the simplest strategy whereby the search moves through the graph of solutions by comparing all of the neighbours in its neighbourhood and accepting the highest-value neighbour. This strategy usually performs poorly as it can easily become stuck in local optima and miss the global optimum. There are several other variants of hill climbing such as stochastic hill climbing which assigns a probability of being accepted to each neighbour that is proportional to the 'steepness' of the move - i.e. how good or bad it is. It then randomly selects a neighbour according to these probabilities so that good moves have a high probability of being selected and weaker moves have a smaller (but feasible) chance of being selected. Stochastic hill climbing may converge more slowly than steepest-descent but it often yields better results. In first-choice hill climbing, rather than generating and evaluating the full neighbourhood, it randomly generates a neighbour until it finds one that improves its current solution. This can be a useful strategy with problems that have a very large neighbourhood. Random restart hill climbing performs several hill-climbing searches starting from different, random starting solutions in the hope that even if each search gets stuck in a local optimum, enough of the solution space has been explored that one of the local optima is close to the global optimum.

\subsection{Simulated Annealing}

The main issue with hill climbing search is that it moves greedily, only accepting a neighbour if it improves upon the current solution. Simulated Annealing \citep{kirkpatrick1983optimization} is a method inspired by annealing in metals where the substance is first melted to allow its atoms to flow freely, and the temperature is slowly lowered until it freezes. The substance spends relatively little time near its melting point and more time close to its freezing point. By cooling it in this way, its atoms are able to move around and find a low-energy state compared to quickly freezing it and inducing defects because the atoms solidified in a sub-optimal arrangement. This process has been adapted to combinatorial optimisation problems where a temperature is cooled during the search according to a cooling schedule. When the temperature is high, the probability of accepting a 'bad' move is high and the probability reduces as the temperature is cooled. This means that the tolerance for moving from a good solution to a worse solution is high to begin with (diversification) but over time, this tolerance reduces and the search focuses on improving the current solution (intensification). In other words, the search starts off more like a random walk of the solution space and transforms into a steepest-descent hill climbing.

\subsection{Tabu Search}

When traversing the solution graph, it is possible that the search will repeatedly visit the same set of solutions. When it enters this kind of cycle, the search stalls and it gets stuck in a local area of the graph. This can happen both when only accepting improving moves or using more complex strategies to accept non-improving moves. Tabu search \citep{glover1989tabu} tries to reduce this problem by using a form of short-term memory to remember recent moves and prevent them from being made again for a given amount of time. If a move is attempted while it is still 'banned', the search must find another move that is not banned. This encourages the search to visit unexplored areas of the solution space and limits getting stuck repeating the same moves. The definition of a move is important because if a move is defined too broadly, it may be encountered too often and the tabu list may stifle the search by banning common moves. Conversely, moves that are too rare will not occur often and the tabu list is redundant since the search may still get caught in a cycle. Similarly, the length of time that a move remains banned for can greatly impact the search's performance by banning too many or too few moves at once.


%quote some papers listed in the survey
%(see \citet{kolisch2001integrated} for a comprehensive survey).

%Slide 22 of local search lecture (Escaping local maxima)