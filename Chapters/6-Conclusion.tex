% Chapter Template

\chapter{Conclusion}
\label{ChapterConclusion}

An implementation of the ASPP was given an integer programming formulation and it was shown that it can find the optimal solution using a complete algorithm. Using warm starts was also shown to improve the time taken to find the optimal solution. The same problem was then modelled in a way that could be solved by a local search algorithm. Several popular metaheuristics such as iterated local search, simulated annealing, and tabu search were blended together to maximise the quality of solutions found by the local search. The results showed that it could find solutions that were optimal or close to optimal, depending on the size and complexity of the problem. The time taken by the local search was of orders of magnitude faster than the complete algorithm. Therefore, it can be concluded that this combination of local search metaheuristics can successfully be applied to the ASPP to provide an fast way for an agile team to plan effectively while also increasing transparency to their stakeholders about the long-term plan of their project.

The contribution of this work over existing work was the approach taken to destroy and repair a solution by focusing on using the natural structure of the dependency graph to remove related assignments during the local search. It shows that while related works have used other approaches such as particle swarm optimisation or genetic algorithms, a combination of iterated local search, simulated annealing and tabu search can give good results.

A final reflection on this thesis (and related work) is that solving the ASPP in this way treats it somewhat like it is a deterministic problem. In reality, agile frameworks are often used in projects precisely because the journey to deliver software is far from deterministic and inherently has many unexpected events and lack of precision in estimates. It may be tempting to think that formalising the problem and methodically solving it can make agile project management less of an art and more of a science, but this was not the aim of this work. It should be viewed as an effort to minimise the operational risks of mismanaging a large software development project so that an agile team can maximise its effectiveness with the estimates it has at any particular time.